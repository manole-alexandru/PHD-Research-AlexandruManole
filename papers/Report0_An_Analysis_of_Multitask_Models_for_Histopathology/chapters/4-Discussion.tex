\chapter{Discussion}
\label{chapter:Discussion}

Multi-task learning approaches have numerous benefits, many being especially useful in the field of Digital Histopathology. Training strategies which allow the encoder to learn from multiple datasets address limitations posed by data scarcity. This yields an increase in performance when compared to single-task models. Moreover, even in setting in which the MTL paradigm is not used to this end, the fusion of two symbiotic objectives results in impressive performances. The MT architecture can also be used to boost the accuracy even when a single problem is tackled by reformulating the problem into multiple objectives. This showcases the versatility of this technique. Lastly, as multi-modal network become more popular, MT networks establish themselves as a worthy candidate to process and mini information from heterogeneous data as prediction heads can be added for each data type as a way to complement the main objective. All these characteristics set MT models apart as a approach which can infer in a manner which is closer to human logic, having a lot of potential in the explainability area as the field further matures. 

On the other hand, the MT Histopathology literature is still in early phases. Most models compare themselves against general architectures, which were not tailored for the approached medical task. The desire to solve multiple problems in the same time, created a need for new datasets which contain annotation for all objectives. This is an extremely costly endeavour, which satisfied the requirements only in few cases. Even if multiple datasets exist in the field, combining them can prove to be cumbersome as the resulting super-set would have samples in which not all tasks have a ground-truth, as most databases only contain annotation for one or a few tasks. 

When it comes to the model themselves, the main limitation is the fact that almost all approaches follow the parallel MT architecture. Further experiments could be conducted in order to evaluate the possible improvements obtained through the transformation into a cross-talk or interactive approach. Vast studies could be conducted in order to search for effective fusion techniques between otherwise independent decoders, especially when two of the objectives are closely knit. 

Lastly, another underdeveloped niche which could prove to be a promising research direction consists of combining supervised objectives with semi-supervised or even unsupervised ones. This would alleviate the need of costly annotation, even if the initial results have little to no chance of being comparable to the potential of two fully supervised ones.